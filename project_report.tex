\documentclass[]{article}

%opening
\title{IoT project: a waste management system}
\author{Giuseppe Leone, 10518770 \\ Alessandro Petocchi, 10661410}
\date{}
\begin{document}
\maketitle
\begin{abstract}
	\noindent Using TinyOS, we provided an implementation, simulated by TOSSIM, of a smart waste management system. This document illustrates the various operation modes, such as trash filling, trash collection by a custom truck mote and, if needed, garbage transfer between bins. 
\end{abstract}

\section{Overview}
The network is composed by two types of motes: the \textit{bin motes}, that collect the generated garbage, and a \textit{truck mote}. The latter is used by the bins to empty their garbage levels, once certain conditions are met.

\noindent An additional capability of the bin motes is the possibility to communicate to its neighbors when they reach full capacity, so that any incoming garbage can be transferred to the nearest available bin, thus avoiding garbage overflow.

\noindent The whole network is emulated using TOSSIM. In the TinyOS script, serial communication is implemented, so that the simulation may be attached to either TOSSIM-Live or NodeRed, where the messages circulating in the network will be visible.
\section{Technical details}
At network startup, specifically when the \texttt{SplitControl.startDone} event is signaled without errors, all motes randomly generate ther X and Y coordinates, that are inserted into a \texttt{mote} struct. 

\noindent When startup is completed, network operation is divided into two phases:
\begin{itemize}
	\item \textbf{Garbage phase:} In this phase the bin motes will collect the trash. This is implemented via a timer called \texttt{TimerTrashThrown}, that is oneshot-started with a random value between 1 and 30 seconds.\\\\ When the timer is fired, a random amount of garbage (1 to 10 units) is generated and added to the bin's \texttt{mote.trash} value. Then the same timer is restarted with a newly generated random value, so that this operation repeats ``periodically".
	\item \textbf{Sensing phase:} The bins check their filling level:
	\begin{enumerate}
		\item $\texttt{mote.trash} < 85$: \textbf{NORMAL MODE}.  
		\item $85 \leq \texttt{mote.trash} < 100$: the bin is almost full. If this condition is met, the bin will go in \textbf{ALERT MODE}.
		\item $\texttt{mote.trash} = 100$: the bin cannot store more garbage, and will go in \textbf{FULL/NEIGHBOR MODE}.
	\end{enumerate}
\end{itemize}
\subsection{Alert mode}
This mode is triggered by setting the \texttt{alertMode} flag to true. If this condition is met, the bin mote will send periodical \texttt{ALERT} type messages to the truck. To accomplish this, we start a 10 second periodical \texttt{TimerAlert}: when it's fired, it will call the \texttt{sendAlertMsg} task. \\\\
The \texttt{alertMsg} type is defined in the header file, and it contains the mote's X and Y coordinates along with its ID number. The message is sent only to the truck mote (identified in the topology with \texttt{TOS\_NODE\_ID} = 8).
\subsection{The truck}
The truck mote has an infinite trash capacity and empties the bin's filling level when requested. Upon the receiving of the \texttt{ALERT} message, the truck will fetch the bin's coordinates, and will set a \texttt{TimerTruck} to simulate the trip from its original position to the bin. \\\\
The travel time is defined as
$$t = \alpha_{truck\rightarrow bin}\sqrt{\displaystyle(x_{truck} - x_{bin})^2 + (y_{truck} - y_{bin})^2}$$
where the multiplicative constant is defined in the header file and is set to 30.  This is needed to differentiate the truck travel time from the trash transfer time between bins in neighbor mode, as we will illustrate later. \\\\
When the timer is fired, the mote will send a \texttt{TRUCK} message to the requesting bin and refresh its coordinates with the ones of the bin.\\Furthermore, if the truck is travelling towards a bin, it will set its \texttt{truckBusy} flag to true, such that other \texttt{ALERT} messages from different bins will be dropped.
\subsection{Neighbor mode}
Lorem ipsum dolor sit amet, consectetur adipiscing elit, sed do eiusmod tempor incididunt ut labore et dolore magna aliqua. Eu lobortis elementum nibh tellus molestie nunc. Faucibus nisl tincidunt eget nullam. Nisl rhoncus mattis rhoncus urna neque viverra justo. Sodales neque sodales ut etiam. Eu non diam phasellus vestibulum lorem sed risus ultricies. Facilisis leo vel fringilla est. Pellentesque habitant morbi tristique senectus et netus. Ultricies lacus sed turpis tincidunt id aliquet. Bibendum neque egestas congue quisque egestas diam in arcu. Massa tincidunt dui ut ornare lectus. Ut lectus arcu bibendum at varius vel. Sodales neque sodales ut etiam. Commodo sed egestas egestas fringilla phasellus faucibus scelerisque. Morbi tristique senectus et netus. Integer eget aliquet nibh praesent tristique magna sit amet.
\section{Simulation and results}
Lorem ipsum dolor sit amet, consectetur adipiscing elit, sed do eiusmod tempor incididunt ut labore et dolore magna aliqua. Eu lobortis elementum nibh tellus molestie nunc. Faucibus nisl tincidunt eget nullam. Nisl rhoncus mattis rhoncus urna neque viverra justo. Sodales neque sodales ut etiam. Eu non diam phasellus vestibulum lorem sed risus ultricies. Facilisis leo vel fringilla est. Pellentesque habitant morbi tristique senectus et netus. Ultricies lacus sed turpis tincidunt id aliquet. Bibendum neque egestas congue quisque egestas diam in arcu. Massa tincidunt dui ut ornare lectus. Ut lectus arcu bibendum at varius vel. Sodales neque sodales ut etiam. Commodo sed egestas egestas fringilla phasellus faucibus scelerisque. Morbi tristique senectus et netus. Integer eget aliquet nibh praesent tristique magna sit amet.
\end{document}
